\section{Цель работы}
Написать программу, в которой создается одномерный числовой массив. После заполнения значениями (случайными числами) массива в нем нужно выполнить циклическую перестановку элементов. Количество позиций для циклической перестановки вводится пользователем с клавиатуры.



\section{Ход работы}

\subsection{Автомат с одним видом товара}

Реализуем алгоритм работы автомата с одним видом товара — напитком за \textbf{9 рублей}.
\begin{itemize}
    \item Начальное состояние — 0. В автомате нет денег. Автомат ожидает внесения монет.
    \item Вносится монета — 1, 2, 5 или 10 рублей. В зависимости от внесённой суммы автомат переходит в следующее состояние, определяемое по таблице переходов (Таблица 1).
    \item В случае, если сумма денег в автомате больше или равна 9 рублей, автомат выдаёт напиток и сдачу и переходит в состояние 0.
    \item В случае, если сумма денег меньше 9, ожидается следующее внесение монеты.
\end{itemize}

Таблица переходов построена по следующему принципу: в первой колонке указано текущее состояние (количество рублей в автомате), во второй колонке — монета, полученная автоматом, в третьей колонке — следующее состояние. В следующих четырёх колонках указан выходной алфавит автомата. Если в колонке стоит единица, то такую сдачу следует выдать.

Таблица переходов представляет собой таблицу переходов для данного конечного автомата.

\begingroup
\renewcommand\arraystretch{0.7}
\begin{longtable}[c]{|c|c|c|c|c|c|c|}
\caption{Таблица переходов для автомата с одним видом товаров}
\\
\hline
State & Insert & Nextstate & Change 1 & Change 2 & Change 2 2 & Change 5 \\ \hline
\endfirsthead
\endhead
0     & 1      & 1         & 0        & 0        & 0          & 0        \\ \hline
0     & 2      & 2         & 0        & 0        & 0          & 0        \\ \hline
0     & 5      & 5         & 0        & 0        & 0          & 0        \\ \hline
0     & 10     & 0         & 1        & 0        & 0          & 0        \\ \hline
1     & 1      & 2         & 0        & 0        & 0          & 0        \\ \hline
1     & 2      & 3         & 0        & 0        & 0          & 0        \\ \hline
1     & 5      & 6         & 0        & 0        & 0          & 0        \\ \hline
1     & 10     & 0         & 0        & 1        & 0          & 0        \\ \hline
2     & 1      & 3         & 0        & 0        & 0          & 0        \\ \hline
2     & 2      & 4         & 0        & 0        & 0          & 0        \\ \hline
2     & 5      & 7         & 0        & 0        & 0          & 0        \\ \hline
2     & 10     & 0         & 1        & 1        & 0          & 0        \\ \hline
3     & 1      & 4         & 0        & 0        & 0          & 0        \\ \hline
3     & 2      & 5         & 0        & 0        & 0          & 0        \\ \hline
3     & 5      & 8         & 0        & 0        & 0          & 0        \\ \hline
3     & 10     & 0         & 0        & 0        & 1          & 0        \\ \hline
4     & 1      & 5         & 0        & 0        & 0          & 0        \\ \hline
4     & 2      & 6         & 0        & 0        & 0          & 0        \\ \hline
4     & 5      & 0         & 0        & 0        & 0          & 0        \\ \hline
4     & 10     & 0         & 0        & 0        & 0          & 1        \\ \hline
5     & 1      & 6         & 0        & 0        & 0          & 0        \\ \hline
5     & 2      & 7         & 0        & 0        & 0          & 0        \\ \hline
5     & 5      & 0         & 1        & 0        & 0          & 0        \\ \hline
5     & 10     & 0         & 1        & 0        & 0          & 1        \\ \hline
6     & 1      & 7         & 0        & 0        & 0          & 0        \\ \hline
6     & 2      & 8         & 0        & 0        & 0          & 0        \\ \hline
6     & 5      & 0         & 0        & 1        & 0          & 0        \\ \hline
6     & 10     & 0         & 0        & 1        & 0          & 1        \\ \hline
7     & 1      & 8         & 0        & 0        & 0          & 0        \\ \hline
7     & 2      & 0         & 1        & 0        & 0          & 0        \\ \hline
7     & 5      & 0         & 1        & 1        & 0          & 0        \\ \hline
7     & 10     & 0         & 1        & 1        & 0          & 1        \\ \hline
8     & 1      & 0         & 0        & 0        & 0          & 0        \\ \hline
8     & 2      & 0         & 1        & 0        & 0          & 0        \\ \hline
8     & 5      & 0         & 0        & 0        & 1          & 0        \\ \hline
8     & 10     & 0         & 0        & 0        & 1          & 1        \\ \hline
\end{longtable}
\endgroup
\FloatBarrier

В листинге 1 приведена реализация данного алгоритма с помощью таблицы переходов.

\begin{lstlisting}[language=c, caption={Реализация первого конечного автомата}]
#include <stdio.h>
#include <stdlib.h>
/*
    * single row of FSM:
    * contains current state, input (a coin),
    * next state and output (the change)
    */
typedef struct row
{
    int state;      // states from 0 to 8
    int insert;     // an inserted coin (1, 2, 5, 10)
    int next_state; // state after coin
    int change[4];  // should a machine return 1, 2, 2x2 or 5 coin of change
} row;
// manually add rows to table
void create_table(row *pointer)
{
    pointer[0] = (struct row){0, 1, 1, {0, 0, 0, 0}};
    pointer[1] = (struct row){0, 2, 2, {0, 0, 0, 0}};
    pointer[2] = (struct row){0, 5, 5, {0, 0, 0, 0}};
    pointer[3] = (struct row){0, 10, 0, {1, 0, 0, 0}};
    pointer[4] = (struct row){1, 1, 2, {0, 0, 0, 0}};
    pointer[5] = (struct row){1, 2, 3, {0, 0, 0, 0}};
    pointer[6] = (struct row){1, 5, 6, {0, 0, 0, 0}};
    pointer[7] = (struct row){1, 10, 0, {0, 1, 0, 0}};
    pointer[8] = (struct row){2, 1, 3, {0, 0, 0, 0}};
    pointer[9] = (struct row){2, 2, 4, {0, 0, 0, 0}};
    pointer[10] = (struct row){2, 5, 7, {0, 0, 0, 0}};
    pointer[11] = (struct row){2, 10, 0, {1, 1, 0, 0}};
    pointer[12] = (struct row){3, 1, 4, {0, 0, 0, 0}};
    pointer[13] = (struct row){3, 2, 5, {0, 0, 0, 0}};
    pointer[14] = (struct row){3, 5, 8, {0, 0, 0, 0}};
    pointer[15] = (struct row){3, 10, 0, {0, 0, 1, 0}};
    pointer[16] = (struct row){4, 1, 5, {0, 0, 0, 0}};
    pointer[17] = (struct row){4, 2, 6, {0, 0, 0, 0}};
    pointer[18] = (struct row){4, 5, 0, {0, 0, 0, 0}};
    pointer[19] = (struct row){4, 10, 0, {0, 0, 0, 1}};
    pointer[20] = (struct row){5, 1, 6, {0, 0, 0, 0}};
    pointer[21] = (struct row){5, 2, 7, {0, 0, 0, 0}};
    pointer[22] = (struct row){5, 5, 0, {1, 0, 0, 0}};
    pointer[23] = (struct row){5, 10, 0, {1, 0, 0, 1}};
    pointer[24] = (struct row){6, 1, 7, {0, 0, 0, 0}};
    pointer[25] = (struct row){6, 2, 8, {0, 0, 0, 0}};
    pointer[26] = (struct row){6, 5, 0, {0, 1, 0, 0}};
    pointer[27] = (struct row){6, 10, 0, {0, 1, 0, 1}};
    pointer[28] = (struct row){7, 1, 8, {0, 0, 0, 0}};
    pointer[29] = (struct row){7, 2, 0, {1, 0, 0, 0}};
    pointer[30] = (struct row){7, 5, 0, {1, 1, 0, 0}};
    pointer[31] = (struct row){7, 10, 0, {1, 1, 0, 1}};
    pointer[32] = (struct row){8, 1, 0, {0, 0, 0, 0}};
    pointer[33] = (struct row){8, 2, 0, {1, 0, 0, 0}};
    pointer[34] = (struct row){8, 5, 0, {0, 0, 1, 0}};
    pointer[35] = (struct row){8, 10, 0, {0, 0, 1, 1}};
}

int main()
{
    // init FSM with precalculated values
    int states_num = 36;
    row *fsm = malloc(sizeof(row) * states_num);
    create_table(fsm);

    // start a machine
    printf("Price is 9 \n\n");

    int cur_state = 0;
    int input;

    while (1)
    {
        // new iteration
        printf("\nCurrent state: %d\n", cur_state);
        printf("Enter coin (1, 2, 5 or 10)? \n");
        scanf("%d", &input);

        if (input != 1 && input != 2 && input != 5 && input != 10)
            continue;  // wrong coin

        int i = 0;
        while (fsm[i].state != cur_state || fsm[i].insert != input)
            i++;  // find current row in table

        if (fsm[i].next_state == 0)  // the machine returned to S0
        {
            // so we give a drink!
            printf("Here's your drink! \n");
            printf("And change: \n");
            printf(" - 1 ruble:  %d \n", fsm[i].change[0]);
            printf(" - 2 rubles: %d \n", fsm[i].change[1]);
            printf(" - 4 rubles: %d \n", fsm[i].change[2]);
            printf(" - 5 rubles: %d \n\n", fsm[i].change[3]);
        }
        // go to next state
        printf("going to state %d \n", fsm[i].next_state);
        cur_state = fsm[i].next_state;
    }
    return 0;
}
\end{lstlisting}

На рисунке 1 приведён граф переходов для данной реализации автомата. На рисунке 2 показана работа программы.

\image{1.png}{Граф переходов первого автомата}{1}
\FloatBarrier

\image{3.png}{Работа конечного автомата}{0.63}
\FloatBarrier


\subsection{Автомат Мили}
Модифицируем автомат для обработки двух видов товаров. Первый товар, как и раньше, стоит 9 рублей, а второй — 3 рубля.
Алгоритм:
\begin{itemize}
    \item Первый шаг алгоритма — выбрать напиток. Автомат находится в состоянии 0, затем в зависимости от выбранного товара — 1 или 2 — автомат переходит в состояние 10 или 20 соответственно.
    \item Далее работа автомата повторяет автомат из первого пункта, только состояния кодируются двумя цифрами, первая из которых обозначает номер напитка.
    \item После выдачи напитка и сдачи автомат возвращается в состояние 0.
\end{itemize}


В листинге 2 приведена реализация данного алгоритма с помощью таблицы переходов.

\begin{lstlisting}[language=c, caption={Фрагмент функции main}]
#include <stdio.h>
#include <stdlib.h>

/*
    * single row of FSM:
    * contains current state, input (a coin),
    * next state and output (the change)
    */
typedef struct row
{
    int state;      // states from 0 to 8
    int insert;     // an inserted coin (1, 2, 5, 10)
    int next_state; // state after coin
    int change[4];  // should a machine return 1, 2, 2x2 or 5 coin of change
    int item;       // an item to give (1 or 2)
} row;

// manually add rows to table
void create_table(row *pointer)
{
    pointer[0] = (struct row){0, 1, 10, {0, 0, 0, 0}, 0};
    pointer[1] = (struct row){0, 2, 20, {0, 0, 0, 0}, 0};
    pointer[2] = (struct row){10, 1, 11, {0, 0, 0, 0}, 0};
    pointer[3] = (struct row){10, 2, 12, {0, 0, 0, 0}, 0};
    pointer[4] = (struct row){10, 5, 15, {0, 0, 0, 0}, 0};
    pointer[5] = (struct row){10, 10, 0, {1, 0, 0, 0}, 1};
    pointer[6] = (struct row){11, 1, 12, {0, 0, 0, 0}, 0};
    pointer[7] = (struct row){11, 2, 13, {0, 0, 0, 0}, 0};
    pointer[8] = (struct row){11, 5, 16, {0, 0, 0, 0}, 0};
    pointer[9] = (struct row){11, 10, 0, {0, 1, 0, 0}, 1};
    pointer[10] = (struct row){12, 1, 13, {0, 0, 0, 0}, 0};
    pointer[11] = (struct row){12, 2, 14, {0, 0, 0, 0}, 0};
    pointer[12] = (struct row){12, 5, 17, {0, 0, 0, 0}, 0};
    pointer[13] = (struct row){12, 10, 0, {1, 1, 0, 0}, 1};
    pointer[14] = (struct row){13, 1, 14, {0, 0, 0, 0}, 0};
    pointer[15] = (struct row){13, 2, 15, {0, 0, 0, 0}, 0};
    pointer[16] = (struct row){13, 5, 18, {0, 0, 0, 0}, 0};
    pointer[17] = (struct row){13, 10, 0, {0, 0, 1, 0}, 1};
    pointer[18] = (struct row){14, 1, 15, {0, 0, 0, 0}, 0};
    pointer[19] = (struct row){14, 2, 16, {0, 0, 0, 0}, 0};
    pointer[20] = (struct row){14, 5, 0, {0, 0, 0, 0}, 1};
    pointer[21] = (struct row){14, 10, 14, {0, 0, 0, 1}, 0};
    pointer[22] = (struct row){15, 1, 16, {0, 0, 0, 0}, 0};
    pointer[23] = (struct row){15, 2, 17, {0, 0, 0, 0}, 0};
    pointer[24] = (struct row){15, 5, 0, {1, 0, 0, 0}, 1};
    pointer[25] = (struct row){15, 10, 0, {1, 0, 0, 1}, 1};
    pointer[26] = (struct row){16, 1, 17, {0, 0, 0, 0}, 0};
    pointer[27] = (struct row){16, 2, 18, {0, 0, 0, 0}, 0};
    pointer[28] = (struct row){16, 5, 0, {0, 1, 0, 0}, 1};
    pointer[29] = (struct row){16, 10, 0, {0, 1, 0, 1}, 1};
    pointer[30] = (struct row){17, 1, 18, {0, 0, 0, 0}, 0};
    pointer[31] = (struct row){17, 2, 0, {1, 0, 0, 0}, 1};
    pointer[32] = (struct row){17, 5, 0, {1, 1, 0, 0}, 1};
    pointer[33] = (struct row){17, 10, 0, {1, 1, 0, 1}, 1};
    pointer[34] = (struct row){18, 1, 0, {0, 0, 0, 0}, 1};
    pointer[35] = (struct row){18, 2, 0, {1, 0, 0, 0}, 1};
    pointer[36] = (struct row){18, 5, 0, {0, 0, 1, 0}, 1};
    pointer[37] = (struct row){18, 10, 0, {0, 0, 1, 1}, 1};
    pointer[38] = (struct row){20, 1, 21, {0, 0, 0, 0}, 0};
    pointer[39] = (struct row){20, 2, 22, {0, 0, 0, 0}, 0};
    pointer[40] = (struct row){20, 5, 0, {0, 1, 0, 0}, 2};
    pointer[41] = (struct row){20, 10, 0, {0, 1, 0, 1}, 2};
    pointer[42] = (struct row){21, 1, 22, {0, 0, 0, 0}, 0};
    pointer[43] = (struct row){21, 2, 0, {0, 0, 0, 0}, 2};
    pointer[44] = (struct row){21, 5, 0, {1, 1, 0, 0}, 2};
    pointer[45] = (struct row){21, 10, 0, {1, 1, 0, 1}, 2};
    pointer[46] = (struct row){22, 1, 0, {0, 0, 0, 0}, 2};
    pointer[47] = (struct row){22, 2, 0, {1, 0, 0, 0}, 2};
    pointer[48] = (struct row){22, 5, 0, {0, 0, 1, 0}, 2};
    pointer[49] = (struct row){22, 10, 0, {1, 1, 1, 1}, 2};
}

int main()
{
    // init FSM with precalculated values
    int states_num = 50;
    row *fsm = malloc(sizeof(row) * states_num);
    create_table(fsm);

    // start a machine
    printf("Price of first item:   9 \n");
    printf("Price of second item:  3 \n\n");

    int cur_state = 0;
    int input;

    while (1)
    {
        // new iteration
        printf("\nCurrent state: %d\n", cur_state);
        if (cur_state == 0)
            printf("Select item (1 or 2)? \n");
        else
            printf("Enter coin (1, 2, 5 or 10)? \n");
        scanf("%d", &input);

        if (cur_state == 0 && input != 1 && input != 2)
            continue; // wrong item

        if (cur_state != 0 && input != 1 && input != 2 && input != 5 && input != 10)
            continue; // wrong coin

        int i = 0;
        while (fsm[i].state != cur_state || fsm[i].insert != input)
            i++; // find current row in table

        if (fsm[i].next_state == 0) // the machine returned to S0
        {
            // so we give a drink!
            printf("Here's your item #%d! \n", fsm[i].item);
            printf("And change: \n");
            printf(" - 1 ruble:  %d \n", fsm[i].change[0]);
            printf(" - 2 rubles: %d \n", fsm[i].change[1]);
            printf(" - 4 rubles: %d \n", fsm[i].change[2]);
            printf(" - 5 rubles: %d \n\n", fsm[i].change[3]);
        }

        // go to next state
        printf("going to state %d \n", fsm[i].next_state);
        cur_state = fsm[i].next_state;
    }

    return 0;
}
    
\end{lstlisting}

На рисунке 3 приведён граф переходов для данной реализации автомата.

\image{2.png}{Граф переходов автомата с двумя товарами}{1}
\FloatBarrier

Таблица переходов построена аналогично. Последняя колонка показывает номер товара, который выдаёт автомат (Таблица 2).


\begingroup
\renewcommand\arraystretch{0.7}
\begin{longtable}[c]{|c|c|c|c|c|c|c|c|}
\caption{Таблица переходов для автомата с двумя видами товаров}
\\
\hline
State & Insert & Nextstate & Change 1 & Change 2 & Change 2 2 & Change 5 & Item \\ \hline
\endfirsthead
%
\endhead
%
0     & 1      & 10        & 0        & 0        & 0          & 0        & 0    \\ \hline
0     & 2      & 20        & 0        & 0        & 0          & 0        & 0    \\ \hline
        &        &           &          &          &            &          &      \\ \hline
10    & 1      & 11        & 0        & 0        & 0          & 0        & 0    \\ \hline
10    & 2      & 12        & 0        & 0        & 0          & 0        & 0    \\ \hline
10    & 5      & 15        & 0        & 0        & 0          & 0        & 0    \\ \hline
10    & 10     & 0         & 1        & 0        & 0          & 0        & 1    \\ \hline
11    & 1      & 12        & 0        & 0        & 0          & 0        & 0    \\ \hline
11    & 2      & 13        & 0        & 0        & 0          & 0        & 0    \\ \hline
11    & 5      & 16        & 0        & 0        & 0          & 0        & 0    \\ \hline
11    & 10     & 0         & 0        & 1        & 0          & 0        & 1    \\ \hline
12    & 1      & 13        & 0        & 0        & 0          & 0        & 0    \\ \hline
12    & 2      & 14        & 0        & 0        & 0          & 0        & 0    \\ \hline
12    & 5      & 17        & 0        & 0        & 0          & 0        & 0    \\ \hline
12    & 10     & 0         & 1        & 1        & 0          & 0        & 1    \\ \hline
13    & 1      & 14        & 0        & 0        & 0          & 0        & 0    \\ \hline
13    & 2      & 15        & 0        & 0        & 0          & 0        & 0    \\ \hline
13    & 5      & 18        & 0        & 0        & 0          & 0        & 0    \\ \hline
13    & 10     & 0         & 0        & 0        & 1          & 0        & 1    \\ \hline
14    & 1      & 15        & 0        & 0        & 0          & 0        & 0    \\ \hline
14    & 2      & 16        & 0        & 0        & 0          & 0        & 0    \\ \hline
14    & 5      & 0         & 0        & 0        & 0          & 0        & 1    \\ \hline
14    & 10     & 14        & 0        & 0        & 0          & 1        & 0    \\ \hline
15    & 1      & 16        & 0        & 0        & 0          & 0        & 0    \\ \hline
15    & 2      & 17        & 0        & 0        & 0          & 0        & 0    \\ \hline
15    & 5      & 0         & 1        & 0        & 0          & 0        & 1    \\ \hline
15    & 10     & 0         & 1        & 0        & 0          & 1        & 1    \\ \hline
16    & 1      & 17        & 0        & 0        & 0          & 0        & 0    \\ \hline
16    & 2      & 18        & 0        & 0        & 0          & 0        & 0    \\ \hline
16    & 5      & 0         & 0        & 1        & 0          & 0        & 1    \\ \hline
16    & 10     & 0         & 0        & 1        & 0          & 1        & 1    \\ \hline
17    & 1      & 18        & 0        & 0        & 0          & 0        & 0    \\ \hline
17    & 2      & 0         & 1        & 0        & 0          & 0        & 1    \\ \hline
17    & 5      & 0         & 1        & 1        & 0          & 0        & 1    \\ \hline
17    & 10     & 0         & 1        & 1        & 0          & 1        & 1    \\ \hline
18    & 1      & 0         & 0        & 0        & 0          & 0        & 1    \\ \hline
18    & 2      & 0         & 1        & 0        & 0          & 0        & 1    \\ \hline
18    & 5      & 0         & 0        & 0        & 1          & 0        & 1    \\ \hline
18    & 10     & 0         & 0        & 0        & 1          & 1        & 1    \\ \hline
        &        &           &          &          &            &          &      \\ \hline
20    & 1      & 21        & 0        & 0        & 0          & 0        & 0    \\ \hline
20    & 2      & 22        & 0        & 0        & 0          & 0        & 0    \\ \hline
20    & 5      & 0         & 0        & 1        & 0          & 0        & 2    \\ \hline
20    & 10     & 0         & 0        & 1        & 0          & 1        & 2    \\ \hline
21    & 1      & 22        & 0        & 0        & 0          & 0        & 0    \\ \hline
21    & 2      & 0         & 0        & 0        & 0          & 0        & 2    \\ \hline
21    & 5      & 0         & 1        & 1        & 0          & 0        & 2    \\ \hline
21    & 10     & 0         & 1        & 1        & 0          & 1        & 2    \\ \hline
22    & 1      & 0         & 0        & 0        & 0          & 0        & 2    \\ \hline
22    & 2      & 0         & 1        & 0        & 0          & 0        & 2    \\ \hline
22    & 5      & 0         & 0        & 0        & 1          & 0        & 2    \\ \hline
22    & 10     & 0         & 1        & 1        & 1          & 1        & 2    \\ \hline
\end{longtable}
\endgroup
\FloatBarrier

На рисунке 4 показана работа программы.

\image{4.png}{Работа конечного автомата}{0.8}
\FloatBarrier

\subsection{Автомат Мура}
Для данного автомата продемонстрируем, как будет выглядеть граф Мура. Особенность автомата Мура заключается в том, что выход автомата определяется только текущим состоянием. Это значит, что информацию о внесённых деньгах приходится хранить в текущих состояниях. Граф представлен на рисунке 5.

\image{5.png}{Автомат Мура}{1}
\FloatBarrier

\section{Выводы о проделанной работе}
В рамках данной работы я познакомился с архитектурой программ на основе конечных автоматов и принципом реализации алгоритма устройства на основе конечных автоматов. Разобрал различия между автоматами Мили и Мура. Реализовал логику работы устройства «Вендинговый автомат» на языке С.
