\section{Цель работы}
Изучить паттерн объектно-ориентированного проектирования «Одиночка» \linebreak(Singleton).

\clearpage


\section{Теоретические сведения}

Для некоторых сущностей важно, чтобы существовал только один экземпляр. Данную функцию выполняет паттерн «Одиночка».

Архитектура паттерна Singleton основана на идее использования глобальной переменной со следующими свойствами:
\begin{enumerate}
    \item Время жизни переменной — от запуска программы до ее завершения.
    \item Предоставляет глобальный доступ, то есть такая переменная может быть доступна из любой части программы. 
\end{enumerate}

Однако использовать глобальную переменную некоторого типа непосредственно невозможно, так как существует проблема обеспечения единственности экземпляра, а именно, возможно создание нескольких переменных того же самого типа.

Для решения этой проблемы паттерн «Одиночка» возлагает контроль над созданием единственного объекта на сам класс. Доступ к этому объекту осуществляется через статическую функцию-член класса, которая возвращает указатель или ссылку на него. Этот объект будет создан только при первом обращении к методу, а все последующие вызовы просто возвращают его адрес. Для обеспечения уникальности объекта конструкторы и оператор присваивания объявляются закрытыми.

{\bf Применимость}
\begin{itemize}
    \item экземпляр некоторого класса существует только в единственном экземпляре, к которому может обратиться любой клиент через известную точку доступа;
    \item единственный экземпляр должен расширяться путем порождения подклассов, а клиенты должны иметь возможность работать с расширенным экземпляром без модификации своего кода.
\end{itemize}


{\bf Реализация}

Паттерн одиночка устроен так, что сам класс гарантирует, что больше одного экземпляра создать не удастся. Чаще всего для этого операция, создающая экземпляры, скрывается за операцией класса (то есть за статической функцией или методом класса), которая гарантирует создание не более одного экземпляра. Обычно такая функция называется \texttt{Instance} или \texttt{GetInstance}. Данная операция имеет доступ к переменной, где хранится уникальный экземпляр, и гарантирует инициализацию переменной этим экземпляром перед возвратом ее клиенту. При таком подходе гарантируется, что «одиночка» будет создан и инициализирован перед первым использованием.
\clearpage



\section{Ход работы}

\subsection{Простой синглтон}
Рассмотрим простую реализацию синглтона. В этой реализации для объекта единожды выделяется область памяти, указатель на которую записывается в статическую переменную-поле класса.

Реализация показана на листинге 1.

\begin{lstlisting}[language=c, caption={Простой синглтон}, lineskip={0pt}]
class Restaurant : public Kitchen
{
    public:
        static Restaurant *getInstance()
        {
            if (!p_instance)
                p_instance = new Restaurant();
            return p_instance;
        }
    ...
\end{lstlisting}

Результат работы программы показан на рисунке 1.

\image{11.png}{Простой синглтон}{1}

Код приведён в приложении.

UML-диаграмма для классов приведена на рисунке 2.
\image{41.png}{UML-диаграмма}{0.4}



\subsection{Синглтон Мейера}
Рассмотрим более продвинутую реализацию сингтона — синглтон Мейера. 

Реализация показана на листинге 2.

\begin{lstlisting}[language=c, caption={Cинглтон Мейера}, lineskip={0pt}]
class Restaurant : public Kitchen
{
public:
    static Restaurant *getInstance()
    {
        static Restaurant inst;
        return &inst;
    }
    ...
\end{lstlisting}

Результат работы программы показан на рисунке 3.

\image{21.png}{Cинглтон Мейера}{1}
Код приведён в приложении.
\FloatBarrier


\subsection{Синглтон в языке Python}
На языке Python синглтон реализовать ещё проще. 

Реализация показана на листинге 3.

\begin{lstlisting}[language=python, caption={Cинглтон в Python}, lineskip={0pt}]
class Restaurant(Kitchen):
    def __new__(cls):
        if not hasattr(cls, "instance"):
            cls.instance = super(Restaurant, cls).__new__(cls)
    ...
\end{lstlisting}

Результат работы программы показан на рисунке 4.

\image{31.png}{Cинглтон в Python}{0.7}
Код приведён в приложении.


\FloatBarrier
\clearpage

\section{Выводы о проделанной работе}
В рамках данной работы изучен паттерн объектно-ориентированного проектирования «Одиночка» (Singleton).

\clearpage