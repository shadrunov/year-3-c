\section{Цель работы}
Программа должна выводить в консоль первые N байт содержимого файла, название которого передано в качестве аргумента командной строки. Число N передаётся в качестве второго аргумента командной строки. Если число N больше количества имеющихся данных, необходимо вывести те данные, которые доступны. Если файл пуст или не существует, необходимо вывести соответствующее сообщение об ошибке и завершить работу программы. 


\section{Ход работы}

\subsection{Описание алгоритма}

Работа программы состоит из нескольких этапов:
\begin{enumerate}
    \item Проверить количество входных аргументов. Если их не два, вывести подсказку.
    \item Сохранить аргументы в переменные path (путь к файлу) и N (число байт).
    \item Вывести полученные параметры в консоль (функция print).
    \item Открыть файл на чтение. Если открыть не удалось (не существует файла), вывести сообщение об ошибке.
    \item Выделить N байт в динамической памяти. В случае ошибки вывести сообщение.
    \item Прочитать N байт из файла. В случае ошибки вывести сообщение. Если прочитано 0 байт, вывести сообщение о том, что файл пуст.
    \item Вывести буфер в консоль. В случае ошибки вывести сообщение.
    \item Освободить память и ресурсы.
\end{enumerate}

Так как программа использует API операционных систем, исходные коды программы для компиляторов gcc и MSVC различаются и приведены в приложении.

\subsection{Компилятор gcc}

При использовании компилятора gcc на Linux мы пользуемся системными вызовами из файла unistd.h: write, open, read, close. Они используются для работы с файлами и консолью.

Для сборки используется команда: \texttt{gcc main.c}

Далее продемонстрируем работу программы (рисунки 1-3).

\image{1.png}{Нет файла}{0.63}
\image{2.png}{Файл пустой}{0.63}
\image{3.png}{Работа программы}{0.63}
\FloatBarrier


\subsection{Компилятор MSVC}

Чтобы запустить эту программу на Windows, нужно заменить системные вызовы на WinAPI. Для этого подключаем файл Windows.h и используем функции WriteConsole, CreateFile, HeapCreate, HeapAlloc, HeapDestroy, ReadFile, CloseHandle.

Для компиляции и сборки программы используем Developer Command Prompt и команду: \texttt{cl win.c}

Далее продемонстрируем работу программы (рисунки 1-3).

\image{4.png}{Нет файла}{0.63}
\image{5.png}{Файл пустой}{0.63}
\image{6.png}{Работа программы}{0.63}
\FloatBarrier


\clearpage

\section{Выводы о проделанной работе}
В рамках данной работы я написал программу, в которой выводятся в консоль первые N байт содержимого файла, название которого передано в качестве аргумента командной строки. Число N передаётся в качестве второго аргумента командной строки. Если число N больше количества имеющихся данных, необходимо вывести те данные, которые доступны. Если файл пуст или не существует, необходимо вывести соответствующее сообщение об ошибке и завершить работу программы. Собрал программу с помощью компилятора gcc и MSVC.

\clearpage