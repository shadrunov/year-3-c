\section{Задание на практическую работу}
Дан массив из 10 слов. Найти сумму остатков от деления каждого из них на 3. Результат поместить в отдельный элемент данных.



\section{Ход работы}
В ходе работы написана программа, решающая поставленную задачу на языке Assembler. В программе используется синтаксис Intel, а также реализован вывод в консоль без подключения библиотек.

\subsection{Структура программы}
В программе присутствуют два сегмента: сегмент данных и сегмент кода. В сегменте данных объявлены массив \textbf{array} из 10 элементов типа \texttt{DW} (definite word, 2 байта) и переменная \textbf{result} типа \texttt{DW}. Начальное значение \textbf{result = 0}, начальное значение массива также задано.

В сегменте кода происходит следующее:
\begin{itemize}
    \item Объявляется цикл из 10 итераций. В каждой итерации текущий элемент массива, начиная с первого, делится на 3 с помощью инструкции \texttt{div}. При этом остаток от деления содержится в части регистра \texttt{ah}. 
    \item После получения остатка в регистр \texttt{ebx} записывается предыдущее значение переменной \texttt{result}, затем командой \texttt{add} прибавляется значение остатка и сумма записывается в переменную \texttt{result}. Значение указателя сдвигается на 2 байта.
    \item Далее вызывается процедура вывода переменной \texttt{result} в консоль, затем выхода из программы.
\end{itemize}

Процедуры \textbf{print} и \textbf{exit} используют системные вызовы (4 — SYS\_WRITE и 1 — SYS\_EXIT) для вывода символов на экран и для выхода из программы.


Пример запуска программы в среде разработки SASM приведен на рисунке 1.
\image{1.png}{Пример запуска программы}{0.86}

\clearpage





\subsection{Работа программы}
Продемонстрируем работу программы с различными входными данными. В качестве входных данных выступает только массив с числами. Будем инициализировать его различными числами. Результат на рисунках 2 — 4.



\image{1.png}{Массив чисел от 1 до 10. Результат — 10}{0.86}
\FloatBarrier

\image{2.png}{Массив из троек. Результат — 0}{0.86}
\FloatBarrier

\image{3.png}{Случайный массив. Результат — 3}{0.86}
\FloatBarrier



\clearpage



\section{Выводы о проделанной работе}
В ходе работы я реализовал программу, подсчитывающую сумму остатков от деления на 3 элементов массива длиной 10. Программа написана на ассемблере. В программе используется синтаксис Intel, а также реализован вывод в консоль без подключения библиотек.