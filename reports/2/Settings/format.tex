%%%%%%%%%%%%%%%%% Оформление ГОСТА%%%%%%%%%%%%%%%%%

% Все параметры указаны в ГОСТЕ на 2021, а именно:

\setdefaultlanguage[spelling=modern]{russian}
\setotherlanguage{english}

\usepackage{fontspec}

\setmainfont{CMU Serif}
\setromanfont{CMU Serif}
\setsansfont{CMU Serif}
% \setmonofont{CMU Typewriter Text}
\setmonofont{Courier New}

% \newfontfamily{\cyrillicfont}{CMU Serif} 
% \newfontfamily{\cyrillicfontrm}{CMU Serif}
\newfontfamily{\cyrillicfonttt}{Courier New Bold}
% \newfontfamily{\cyrillicfontsf}{CMU Serif}

\defaultfontfeatures{Ligatures=TeX}
\setmainfont[Ligatures=TeX]{CMU Serif}

\newenvironment{Times}
{
    \setmainfont{Times New Roman}
}
{
    \setmainfont{CMU Serif}
}



% шрифт для URL-ссылок
\urlstyle{same}

% Междустрочный интервал должен быть равен 1.5 сантиметра.
\linespread{1.5} % междустрочный интервал

% Каждая новая строка должна начинаться с отступа равного 1.25 сантиметра.
\setlength{\parindent}{1.25cm} % отступ для абзаца

% Путь до папки с изображениями
\graphicspath{ {./Images/} }

% Внесение titlepage в учёт счётчика страниц
\makeatletter
\renewenvironment{titlepage} {
    \thispagestyle{empty}
}


% Цвет гиперссылок и цитирования
\usepackage{hyperref}
\hypersetup{
    colorlinks=true,
    linkcolor=black,
    filecolor=blue,
    citecolor = black,
    urlcolor=blue,
}

\usepackage[section]{placeins}
\captionsetup[figure]{labelformat=simple, labelsep=endash}
\captionsetup[table]{labelsep=endash, justification=raggedright, singlelinecheck=off}
\captionsetup[lstlisting]{labelformat=simple, labelsep=endash, font=sf}


\usepackage{titlesec}
\titleformat{\section}{\bfseries}{\thesection}{1em}{}
\titleformat{\subsection}{\bfseries}{\thesubsection}{1em}{}
\titleformat{\subsubsection}{\bfseries}{\thesubsubsection}{1em}{}

\renewcommand{\labelitemi}{-}

\usepackage{etoolbox}
\makeatletter
\patchcmd{\ps@plain}{\thepage}{\sffamily\thepage}{}{}
\makeatother
\pagestyle{plain} % execute this command to load the new settings