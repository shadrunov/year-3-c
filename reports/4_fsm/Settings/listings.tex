\usepackage{listings} % библиотека листингов
\usepackage{color} % подсветка листинга


\definecolor{mygreen}{rgb}{0,0.6,0}
\definecolor{mymauve}{rgb}{0.58,0,0.82}
\definecolor{backcolour}{rgb}{0.95,0.95,0.92}
\definecolor{codegray}{rgb}{0.5,0.5,0.5}

\lstset{ % Подробнее про настройку листингов https://en.wikibooks.org/wiki/LaTeX/Source_Code_Listings
  backgroundcolor=\color{white},   % цвет фона
  basicstyle=\ttfamily\footnotesize,    % семейсто, размер шрифта
  breakatwhitespace=true,               % разрыв строк только при пробеле
  breaklines=true,                      % перенос строк
  captionpos=b,                         % месторасположение подписи bottom
  commentstyle=\color{mygreen},         % цвет комментария (не распростроняется на кириллицу)
  keepspaces=true,                      %
  keywordstyle=\color{blue},            %
  showspaces=false,                     % отключена замена пробелов на нижние подчеркивания
  showstringspaces=false,               % отключена замена пробелов на нижние подчеркивания
  showtabs=false,                       % отключена замена табуляций на нижние подчеркивания
  stepnumber=1,                         %
  stringstyle=\color{mymauve},          % цвет литералов
  tabsize=4,
  % numberstyle=\tiny\color{codegray}
  numberstyle=\small\color{codegray},
  numbers=left,
  numbersep=5pt,
  lineskip={-5pt}
}

% Подписи к листингам на русском языке.
\renewcommand\lstlistingname{Листинг}
\renewcommand\lstlistlistingname{Листинги}



