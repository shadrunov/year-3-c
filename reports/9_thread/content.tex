\section{Цель работы}
В файле записан ряд целых чисел, разделённых пробелом. Программа должна считать имя файла из первого аргумента командной строки и рассчитать сумму квадратов записанных в файл чисел. Для расчёта суммы квадратов программа должна создать N потоков (N передаётся вторым аргументом командной строки) и передать каждому из них часть полученных чисел. Каждый из потоков должен рассчитать сумму квадратов переданных ему чисел и вернуть её родительскому. Родительский процесс должен просуммировать полученные от дочерних числа и вывести на консоль итоговую сумму. Если исходный файл не существует, или в нём записано менее двух чисел, следует вывести соответствующее сообщение для пользователя и завершить работу программы. 


\section{Ход работы}

\subsection{Описание алгоритма}
\subsubsection{Основной процесс}

Работа программ (main) состоит из нескольких этапов:
\begin{enumerate}
    \item Проверить количество входных аргументов. Если их не два, вывести подсказку.
    \item Сохранить аргументы в переменные input\_path (путь к файлу) и N (число байт).
    \item Вывести полученные параметры в консоль (функция printf).
    \item Открыть файл на чтение. Если открыть не удалось (не существует файла), вывести сообщение об ошибке.
    \item Считать файл слово за словом, считая количество входных чисел M (для чтения используем небольшой буфер и функцию fscanf). Полученное количество вывести на экран.
    \item Если количество меньше 2, вывести сообщение об ошибке.
    \item Если количество потоков N больше M / 2, уменьшить N до M / 2.
    \item Вычислить количество данных для каждого потока ($M / N$ для всех, кроме последнего, и $n + M \% N$ для последнего), вывести на экран.
    \item Для каждого потока в цикле:
    \begin{itemize}
        \item Записать в массив номер потока, количество чисел и числа
        \item Запустить новый поток, передать массив в качестве аргумента.
        \item Поток рассчитывает сумму квадратов, блокирует мьютекс и обновляет глобальную переменную.
    \end{itemize}
    \item Закрыть входной файл.
    \item Распечатать результат — глобальную переменную.
\end{enumerate}


Программа использует C Standart Library для работы с файлами и OS API для работы с потоками, поэтому эта часть отличается для Linux и Windows.

\subsection{Компилятор gcc}

При использовании компилятора gcc на Linux мы пользуемся библиотекой \texttt{pthread.h}. Исходный код программы для Linux приведён в приложении.

Для сборки основной программы используется команда: \texttt{gcc main.c -o main}. 
Продемонстрируем работу программы (рисунки 1-4).

\image{11.png}{Неверные аргументы}{0.7}
\image{12.png}{Входной файл не существует}{0.7}
\image{13.png}{Входной файл пустой}{0.7}
\image{14.png}{Работа программы}{1}
\FloatBarrier

Потоки отображаются в htop (рисунок 5). 

\image{15.png}{Потоки}{1}
\FloatBarrier
\clearpage


\subsection{Компилятор MSVC}

Чтобы запустить эту программу на Windows, нужно заменить системные вызовы на WinAPI. Для этого подключаем файл Windows.h и используем функции CreateThread, ResumeThread, WaitForSingleObject. Исходный код программы для Windows приведён в приложении.

Для компиляции и сборки программы используем Developer Command Prompt и команду: \texttt{cl main.c /link /out:main.exe}

Далее продемонстрируем работу программы (рисунки 6-9).

\image{21.png}{Неверные аргументы}{0.6}
\image{22.png}{Входной файл не существует}{0.6}
\image{23.png}{Входной файл пустой}{0.7}
\image{24.png}{Работа программы}{0.86}
\FloatBarrier

Потоки отображаются в Process Hacker (рисунок 10).

\image{25.png}{Потоки}{0.9}
\FloatBarrier
\clearpage


\subsection{Python}

Аналогичный функционал присутствует в языке Python. Код программы на Python приведён в приложении. 

Компиляция не предусмотрена и запуск осуществляется командой: \texttt{python main.py input.txt 3}.

Продемонстрируем работу программы (рисунки 11-13).

\image{31.png}{Неверные аргументы}{0.7}
\image{32.png}{Входной файл не существует}{0.7}
\image{33.png}{Работа программы}{0.86}
\FloatBarrier

Потоки отображаются в htop (рисунки 12-13).

\image{34.png}{Потоки}{1}
\FloatBarrier
\clearpage

\section{Выводы о проделанной работе}
В рамках данной работы я написал программу, которая считывает имя файла из первого аргумента командной строки и рассчитывает сумму квадратов записанных в файл чисел. Для расчёта суммы квадратов программа должна создать N потоков (N передаётся вторым аргументом командной строки) и передать каждому из них часть полученных чисел. Каждый из потоков должен рассчитать сумму квадратов переданных ему чисел и вернуть её родительскому. Родительский поток должен просуммировать полученные от дочерних числа и вывести на консоль итоговую сумму. Если исходный файл не существует, или в нём записано менее 2 чисел, следует вывести соответствующее сообщение для пользователя и завершить работу программы. Скомпилировал программу с помощью компиляторов gcc и MSVC, а также реализовал аналогичный функционал на языке Python.

\clearpage